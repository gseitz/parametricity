\begin{frame}[fragile]
\frametitle{Scaling Parametricity}
\begin{lstlisting}[style=scala]
def forallM[F[_]: Monad]
  (p: A => F[Boolean], o: Option[A]): F[Boolean]
\end{lstlisting}
\begin{theorem}
  The \lstinline{Boolean} result depends on zero or more of
  \begin{itemize}
    \item Whether the \lstinline{Option} is a \lstinline{Some} or \lstinline{None}.
    \item If the \lstinline{Option} is a \lstinline{Some}, then the result of having applied the given function to the \lstinline{Some} value.
  \end{itemize}
\end{theorem}
\end{frame}

\begin{frame}[fragile]
\frametitle{Scaling Parametricity}
  We can conclude that there are 8 possible inhabitants \ref{forallM}:
  \begin{enumerate}
    \item always \lstinline{false}
    \item always \lstinline{true}
    \item \lstinline{o.isDefined}
    \item \lstinline{o.isEmpty}
    \item \lstinline{Some(a) => p(a) else false}
    \item \lstinline{Some(a) => p(a) else true}
    \item \lstinline{Some(a) => !p(a) else false}
    \item \lstinline{Some(a) => !p(a) else true}
  \end{enumerate}
\end{frame}

\begin{frame}[fragile]
\frametitle{Scaling Parametricity}
  \begin{block}{Importantly}
  The implementation may only use the monad primitive operations, even though the use-case may apply a specific monad context.
  \end{block}
\end{frame}

\begin{frame}[fragile]
\frametitle{Scaling Parametricity}
  \begin{block}{For example}
  The \lstinline{forallM} function definitely does not perform any IO effects, even though the function user may apply that specific use-case.
  \end{block}
  and so on \ldots
\end{frame}
